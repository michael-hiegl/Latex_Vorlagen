\documentclass[11pt]{report}

\usepackage[a4paper, top=2.5cm, bottom=2.5cm, left=3cm, right=2cm]{geometry}
\usepackage[ngerman]{babel}
\usepackage[utf8]{inputenc}
\usepackage[T1]{fontenc}
\usepackage{helvet}
\renewcommand{\familydefault}{\sfdefault}
\usepackage{setspace}
\usepackage{titlesec}
\usepackage{tocloft}
\usepackage{chngcntr}
\usepackage{fancyhdr}
\usepackage{subcaption}
\usepackage{graphicx}
\usepackage{float}
\usepackage{siunitx}
\usepackage{multicol}
\usepackage{tabularx}
\graphicspath{ {./Bilder/} }

% Schriftgrößen und Zeilenabstand
\renewcommand{\normalsize}{\fontsize{11pt}{\baselineskip}\selectfont}
\titleformat{\chapter}{\fontsize{14pt}{\baselineskip}\bfseries}{\thechapter}{1em}{}
\titlespacing*{\chapter}{0pt}{-20pt}{20pt}
\setlength{\cftchapnumwidth}{2.5em}
\setlength{\cftsecnumwidth}{3.5em}
\setlength{\cftsubsecnumwidth}{4.5em}
\setlength{\cftsubsubsecnumwidth}{5.5em}
\counterwithout{footnote}{chapter}
\setlength{\parindent}{0pt}

% Seitenstil
\fancypagestyle{plain}{
    \fancyhf{}
    \renewcommand{\headrulewidth}{0pt}
    \fancyfoot[R]{\thepage}
}
\pagestyle{plain}

%% Formatierung des Inhaltsverzeichnisses
%\renewcommand{\cftdot}{}
%\renewcommand{\cftchapleader}{\cftdotfill{\cftdotsep}}
%\renewcommand{\cftchapfont}{\normalsize\bfseries}
%\renewcommand{\cftsecfont}{\normalsize}
%\renewcommand{\cftsubsecfont}{\normalsize}
%\renewcommand{\cftsubsubsecfont}{\normalsize}

\begin{document}

% Titelseite
\begin{titlepage}
    \centering
    \includegraphics[width=0.4\textwidth]{Technische_Hochschule_Deggendorf_logo.png}\par\vspace{1cm}
    {\scshape\LARGE Technische Hochschule Deggendorf \par}
    \vspace{1cm}
    {\Large Fakultät angewandte Naturwissenschaften und Wirtschaftsingenieurwesen\par}
    \vspace{1cm}
    {\Large Studiengang Mechatronik – Schwerpunkt digitale Produktion\par}
    \vspace{1cm}
    {\scshape\Large Bachelorarbeit\par}
    \vspace{1.5cm}
    {\huge\bfseries Titel der Bachelorarbeit\par}
    \vspace{2cm}
    {\Large\itshape Michael Hiegl\par}
    \vfill
    Betreut von\par
    Prof. Dr. Vorname Nachname

    \vfill

    % Unterer Teil der Seite
    {\large Abgabedatum: \today\par}
\end{titlepage}

% Seitennummerierung in römischen Zahlen für den Anfang der Arbeit
\pagenumbering{Roman}
\setcounter{page}{1} % Seitenzahl manuell auf 1 setzen

% Inhaltsverzeichnis
\cleardoublepage
\phantomsection
\addtocontents{toc}{\protect\thispagestyle{empty}} % Seitenzahl im Inhaltsverzeichnis entfernen
\tableofcontents
\newpage

% Seitennummerierung in arabischen Zahlen für den Hauptteil der Arbeit
\pagenumbering{arabic}

% Kapitelüberschriften formatieren
\titleformat{\chapter}{\fontsize{14pt}{\baselineskip}\bfseries}{\thechapter}{1em}{}

% Kapitel 1
\chapter{Einleitung}
Lorem ipsum dolor sit amet, consetetur sadipscing elitr, sed diam nonumy eirmod tempor invidunt ut labore et dolore magna aliquyam erat, sed diam voluptua. At vero eos et accusam et justo duo dolores et ea rebum. Stet clita kasd gubergren, no sea takimata sanctus est Lorem ipsum dolor sit amet. Lorem ipsum dolor sit amet, consetetur sadipscing elitr, sed diam nonumy eirmod tempor invidunt ut labore et dolore magna aliquyam erat, sed diam voluptua. At vero eos et accusam et justo duo dolores et ea rebum. Stet clita kasd gubergren, no sea takimata sanctus est Lorem ipsum dolor sit amet. \newline \newline
Lorem ipsum dolor sit amet, consetetur sadipscing elitr, sed diam nonumy eirmod tempor invidunt ut labore et dolore magna aliquyam erat, sed diam voluptua. At vero eos et accusam et justo duo dolores et ea rebum. Stet clita kasd gubergren, no sea takimata sanctus est Lorem ipsum dolor sit amet. Lorem ipsum dolor sit amet, consetetur sadipscing elitr, sed diam nonumy eirmod tempor invidunt ut labore et dolore magna aliquyam erat, sed diam voluptua. At vero eos et accusam et justo duo dolores et ea rebum. Stet clita kasd gubergren, no sea takimata sanctus est Lorem ipsum dolor sit amet.

\begin{figure}[h]
\includegraphics[scale=0.05]{Apfel.jpg}
\centering
\caption{Ein Apfel}
\end{figure}

% Kapitel 2
\chapter{Hauptteil}
\section{Abschnitt 1}
Text...

\begin{figure}[h]
\includegraphics[scale=0.05]{Apfel.jpg}
\centering
\caption{Ein Apfel}
\end{figure}

\section{Abschnitt 2}
Text...

\begin{figure}[h]
\includegraphics[scale=0.05]{Apfel.jpg}
\centering
\caption{Ein Apfel}
\end{figure}

% Schlusskapitel
\chapter{Schlussfolgerung}
Text...

\begin{figure}[h]
\centering
\begin{subfigure}{0.49\textwidth}
\centering
\includegraphics[width = \textwidth]{Apfel.jpg}
\caption{Ein Apfel}
\label{fig:left}
\end{subfigure}
\begin{subfigure}{0.49\textwidth}
\centering
\includegraphics[width = \textwidth]{Orange.jpg}
\caption{Eine Orange}
\label{fig:right}
\end{subfigure}
\caption{Beide Früchte}
\end{figure}

\chapter{Formelzeichenverzeichnis}
\begin{itemize}
	\item 1 Bedeutung des Zeichen 1
	\item 2 Bedeutung des Zeichen 2
\end{itemize}

\chapter{Abkürzungsverzeichnis}
\begin{itemize}
	\item 1 Abkürzung 1
	\item 2 Abkürzung 2
\end{itemize}

\chapter{Literaturverzeichnis}
\begin{itemize}
	\item 1 Leschik, M.: Word für Windows 6.0, Wissenschaftlich Arbeiten, optimal. 2. Aufl. Koschenbroich, bhv-Verlag, 1994.
	\item 2 Leschik, M.: Word für Windows 6.0, Wissenschaftlich Arbeiten, optimal. 2. Aufl. Koschenbroich, bhv-Verlag, 1994.
\end{itemize}

\chapter{Erklärung}
\underline{Name des Studierenden:}\\
22109460 Michael Hiegl\\

Ich erkläre hiermit, dass ich die Arbeit selbstständig und ohne fremde Hilfe verfasst, noch nicht anderweitig für Prüfungszwecke vorgelegt, keine anderen als die angegeben Quellen oder Hilfsmittel benutzt sowie wörtliche und sinngemäße Zitate als solche gekennzeichnet habe.\\\\

93413 Cham, den \today\\\\
\underline{Unterschrift:}\\
\includegraphics[scale=0.3]{Signatur.png}\\

%% Formelzeichen- oder Abkürzungsverzeichnis
%\chapter{Formelzeichen- oder Abkürzungsverzeichnis}
%\addcontentsline{toc}{chapter}{Formelzeichen- oder Abkürzungsverzeichnis}
%\begin{itemize}
%    \item 1 Bedeutung der Abkürzung 1
%    \item 2 Bedeutung der Abkürzung 2
%\end{itemize}
%
%% Literaturverzeichnis
%\cleardoublepage
%\phantomsection
%\addcontentsline{toc}{chapter}{Literaturverzeichnis}
%\chapter{Literaturverzeichnis}
%\begin{itemize}
%    \item 1 Autorenname, Buchtitel, Verlagsort, Verlag, Jahr.
%    \item 2 Autorenname, Artikelname, Zeitschrift, Band, Seiten, Jahr.
%\end{itemize}

\end{document}
